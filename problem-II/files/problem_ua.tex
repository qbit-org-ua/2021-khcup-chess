\begin{problem}{}{}{}{0.2 секунди}{64 мегабайта}
% Друга шахова партія з Біллом

\begin{wrapfigure}{r}{0.35\textwidth}
\vspace{-20pt}
  \begin{center}
    \includegraphics[width=0.35\textwidth,natwidth=232,natheight=217]{pic.jpg}
  \end{center}
  \vspace{-20pt}
  \vspace{1pt}
\end{wrapfigure}

В цій задачі належить битися в шахи з самим Біллом Сайфером.
На перший погляд, у вас виграшна позиція $-$ білий король і ферзь проти одинокого короля Білла.
Але якщо ви зробите некоректний хід, або не виграєте за 50 ходів, або поставите пат, то ви програли!
Крім того, в цій партії  вам не можна ходити королем, але гарантується, що виграти можна.

\InputFile
З початку ваша програма повинна прочитати із стандартного потоку введення  поточну позицію гри:
в тьох окремих рядках в стандартній шаховій нотації записані позиції білого короля, белого ферзя и чорного короля.
Гарантується, що позиція коректна, тобто жодна з фігур не перебуває під боєм.
Гарантується, що білий король стоїть в одній з позицій: {\t{c3}}, {\t{c6}}, {\t{f3}} або {\t{f6}}.

\OutputFile
Ваша програма повинна зробити перший хід за білих та вивести його в стандартній шаховій нотації (наприклад, {\t{Kf3}}).
Гарантується, що тестуюча система буде робити коректний хід.
Далі в кожному новому рядку вхідного потоку даних тестуюча  система буде повідомляти хід чорного короля 
також в стандартній шаховій нотації. На кожний хід тестуючої системи ви повинні відповісти ходом білого ферзя (наприклад, {\t{Qg1}}). 
Якщо ваша програма своїм черговим ходом оголошує мат, то до виведення необхідно додати символ {\t{\#}}
(наприклад, {\t{Qg7\#}}) та  закінчити роботу програми.

Якщо ваша програма виведе  некоректний хід, ви отримаєте вердикт {\t{PE}}; якщо ваша програма 
не поставить мат чорному королю до 50 ходу включно, ви отримаєте  вердикт {\t{WA}}.

\Example
\begin{example}
\exmp{f6
b3
e8
\bigskip
\includegraphics[width=0.95\textwidth,natwidth=232,natheight=217]{pos.png}
}{[your solution] Qd1
[test system] Kf8
[your solution] Qd7 
[test system] Kg8
[your solution] Qg7\#
}%
\end{example}

\Note
Задача не має однозначної відповіді,  оцінювання буде здійснено за результатами декількох ігор,
тому варіант відповіді  $-$ це тільки один із сотен можливих  варіантів.

\end{problem}

