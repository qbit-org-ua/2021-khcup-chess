\begin{problem}{}{}{}{0.2 секунды}{64 мегабайта}
% Первая партия в шахматы с Биллом

\begin{wrapfigure}{r}{0.35\textwidth}
\vspace{-20pt}
  \begin{center}
    \includegraphics[width=0.35\textwidth,natwidth=232,natheight=217]{pic.jpg}
  \end{center}
  \vspace{-20pt}
  \vspace{1pt}
\end{wrapfigure}

В этой задаче вам предстоит сразиться в шахматы с самим Биллом Сайфером.
На первый взгляд, у вас всё хорошо $-$ белый король и ферзь против одинокого короля Билла.
Но если вы сделаете некорректный ход или не выиграете за 50 ходов или поставите пат, то вы проиграли!




\InputFile
В самом начале ваша программа должна прочитать со стандартного потока ввода текущую позицию игры:
в трёх отдельных строках записаны в стандартной шахматной нотации позиции белого короля, белого ферзя и чёрного короля.
Гарантируется, что позиция корректная, то есть ни одна из фигур не находится под боем.

\OutputFile
Ваша программа должна сделать первый ход за белых и вывести его в стандартной шахматной нотации (например, {\t{Kf3}}).
Гарантируется, что тестирущая система будет делать корректный ход.
Далее, в каждой новой строке входного потока данных тестирующая система будет сообщать ход чёрного короля 
также в стандартной шахматной нотации. На каждый ход тестирующей системы вы должны ответить ходом белого ферзя (например, {\t{Qg1}})
или белого короля (например, {\t{Kc3}}). Если ваша программа своим очередным ходом объявляет мат, то к выводу надо добавить символ {\t{\#}}
(например, {\t{Qg7\#}}) и завершить работу программы.

Если ваша программа выведет некорректный ход, вы получите вердикт {\t{PE}}; если ваша программа 
не поставите мат чёрному королю до 50 хода включительно, вы получите вердикт {\t{WA}}.

\if 0
1. Фd4 Kpf7 2. Фd6 Kpg7 3. Фe6 Kpf8 4. Фd7 Kpg8 5. Kpb2 Kpf8 6. Kpc3 Kpg8 7. Kpd4 Kpf8 8. Kpe5 Kpg8 9. Kpf6 Kph8 10. Фg7#
\fi

\Example
\begin{example}
\exmp{a1
b2
e6
\bigskip
\includegraphics[width=0.95\textwidth,natwidth=232,natheight=217]{pos.png}
}{[your solution] Qd4
[test system] Kf7
[your solution] Qd6 
[test system] Kg7
[your solution] Qe6
[test system] Kf8
[your solution] Qd7
[test system] Kg8
[your solution] Kb2
[test system] Kf8
[your solution] Kc3
[test system] Kg8
[your solution] Kd4
[test system] Kf8
[your solution] Ke5
[test system] Kg8
[your solution] Kf6
[test system] Kh8
[your solution] Qg7\#
}%
\end{example}

\Note
Задача не имеет однозначного ответа и оценивание будет проводиться по итогу нескольких игр,
поэтому вариант ответа $-$ это лишь один из сотен возможных вариантов.

Если вы не знаете, как ходят шахматные фигуры, что такое <<мат>> или <<пат>>, можете прочитать про это в интернете. :) 
\end{problem}

