\begin{problem}{}{}{}{0.2 секунди}{64 мегабайта}
% Перша шахова партія з Білом

\begin{wrapfigure}{r}{0.35\textwidth}
\vspace{-20pt}
  \begin{center}
    \includegraphics[width=0.35\textwidth,natwidth=232,natheight=217]{pic.jpg}
  \end{center}
  \vspace{-20pt}
  \vspace{1pt}
\end{wrapfigure}

В цій задачі вам належить битися в шахи з самим Біллом Сайфером.
На перший погляд, у вас все добре $-$ білий король і ферзь проти самотнього короля Білла.
Але якщо ви зробите некоректний хід, або не виграєте за 50 ходів, або поставите пат, то ви програли!




\InputFile
З початку ваша програма повинна прочитати із стандартного потоку введення поточну позицію гри:
в трьох окремих рядках в стандартній шаховій нотації записані позиції білого короля, білого ферзя та чорного короля.
Гарантується, що позиція коректна, тобто жодна з фігур не перебуває під боєм.

\OutputFile
Ваша програма повинна зробити перший хід за білих і вивести його в стандартній шаховій нотації (наприклад, {\t{Kf3}}).
Гарантується, що тестуюча система буде робити коректний ход.
Далі в кажному новому рядку вхідного потока даних тестуюча система буде повідомляти хід чорного короля також в стандартній шаховій нотації. На кожен хід тестуючої  системи ви повинні  відповісти ходом білого ферзя (наприклад, {\t{Qg1}})
або білого короля (наприклад, {\t{Kc3}}). Якщо ваша програма своїм черговим ходом оголошує мат, то до виведення треба додати символ {\t{\#}}
(наприклад, {\t{Qg7\#}}) та завершити роботу програми.

Якщо ваша програма виконує некоректний хід, ви отримаєте вердикт {\t{PE}}; якщо ваша програма 
не поставить мат чорному королю до 50 ходу включно, ви отримаєте вердикт {\t{WA}}.

\if 0
1. Фd4 Kpf7 2. Фd6 Kpg7 3. Фe6 Kpf8 4. Фd7 Kpg8 5. Kpb2 Kpf8 6. Kpc3 Kpg8 7. Kpd4 Kpf8 8. Kpe5 Kpg8 9. Kpf6 Kph8 10. Фg7#
\fi

\Example
\begin{example}
\exmp{a1
b2
e6
\bigskip
\includegraphics[width=0.95\textwidth,natwidth=232,natheight=217]{pos.png}
}{[your solution] Qd4
[test system] Kf7
[your solution] Qd6 
[test system] Kg7
[your solution] Qe6
[test system] Kf8
[your solution] Qd7
[test system] Kg8
[your solution] Kb2
[test system] Kf8
[your solution] Kc3
[test system] Kg8
[your solution] Kd4
[test system] Kf8
[your solution] Ke5
[test system] Kg8
[your solution] Kf6
[test system] Kh8
[your solution] Qg7\#
}%
\end{example}

\Note
Задача не має однозначної відповіді,  оцінювання буде здійснено за результатами декількох ігор,
тому варіант відповіді  $-$ це тільки один із сотен можливих  варіантів.

\end{problem}

